\section{Умные указатели}

Рассмотрим следующий код: 

\begin{minted}[
linenos,
frame=lines,
framesep=2mm]
{c++}
container* create_container()
{
    container* c = new container();
    fill(*c);
    return c;
}
\end{minted}

В приведенном коде, при возникновении исключения в функции fill, это исключение пролетит наружу функции create\_container. Однако выделенный с помощью new контейнер c не будет освобожден. Возможным способом исправления этой ошибки является использование try…catch блока:

\begin{minted}[
linenos,
frame=lines,
framesep=2mm]
{c++}
container* create_container()
{
    container* c = new container();	
    try
    {
        fill(*c);
    }
    catch(...)
    {
        delete c;
        throw;
    }
    return c;
}
\end{minted}

Такой способ применим если лишь к простейшим функциям. При начилии нескольких объектов, которые требуется удалить или в присутствии сложного control-flow такое использование try...catch становится непрактичным.

Для решения этой проблемы в C++11 появились классы умных указателей (\mintinline{c++}{std::unique_ptr}, \mintinline{c++}{std::shared_ptr} и \mintinline{c++}{std::weak_ptr}). Эти классы являются RAII-обертками над обычными указателями, которые в своём деструкторе делают \mintinline{c++}{delete} тому объекту, на который они ссылаются. При использовании \mintinline{c++}{std::unique_ptr} приведенный выше (корректный) код может быть записан следующим образом:

\begin{minted}[
linenos,
frame=lines,
framesep=2mm]
{c++}
std::unique_ptr<container> create_container()
{
    std::unique_ptr<container> c(new container());
    fill(*c);
    return c;
}
\end{minted}

\subsection{\mintinline{c++}{std::unique_ptr}}
Самым простым умным указателем является \mintinline{c++}{std::unique_ptr}. 
Внутри себя \mintinline{c++}{std::unique_ptr} хранит один указатель \mintinline{c++}{T* ptr} и делает ему \mintinline{c++}{delete} в дескрукторе.

\begin{minted}[
linenos,
frame=lines,
framesep=2mm]
{c++}
template<class T>
class unique_ptr
{
private:
    T* ptr;
public:
    ~unique_ptr()
    {
        delete ptr;
    }
    ...
}
\end{minted}

\mintinline{c++}{std::unique_ptr} имеет операторы \mintinline{c++}{*} и \mintinline{c++}{->}, поэтому им можно пользоваться как обыным указателем:

\begin{minted}[
linenos,
frame=lines,
framesep=2mm]
{c++}
T& operator*() const { return *ptr; }
T* operator->() const noexcept { return ptr; }
\end{minted}

\mintinline{c++}{std::unique_ptr} имеет следующие функции:

\mintinline{c++}{get()} --- возвращает ptr, хранящийся внутри. \mintinline{c++}{get()} может использоваться если необходимо передать в некоторую функцию сырой указатель на объект, а имеется \mintinline{c++}{unique_ptr} на него.

\begin{minted}[
linenos,
frame=lines,
framesep=2mm]
{c++}
T* get() const { return ptr; }
\end{minted}

\mintinline{c++}{release()} --- зануляет ptr, хранящийся внутри, а старое значение возвращает наружу. \mintinline{c++}{release()} может использоваться если необходимо передать в некоторую функцию сырой указатель на объект и известно, что эта функция самостоятельно удалит переданный объект.

\begin{minted}[
linenos,
frame=lines,
framesep=2mm]
{c++}
T* release()
{
    T* tmp = ptr;
    ptr = nullptr;
    return tmp;
}
\end{minted}

\mintinline{c++}{reset(p)} - заменяет ptr, хранящийся внутри, на p, и делает \mintinline{c++}{delete} старому ptr.
\begin{minted}[
linenos,
frame=lines,
framesep=2mm]
{c++}
void reset(T* p)
{
    delete ptr;
    ptr = p;
}
\end{minted}

Оператор присваивания и конструктор копирования у \mintinline{c++}{unique_ptr} явно запрещены:
\begin{minted}[
linenos,
frame=lines,
framesep=2mm]
{c++}
unique_ptr(unique_ptr& other) = delete;
unique_ptr& operator=(unique_ptr& other) = delete;
\end{minted}
При попытке скопировать или присвоить \mintinline{c++}{unique_ptr} выдаётся ошибка на этапе компиляции.

\mintinline{c++}{unique_ptr} имеет move-оператор присваивания и move-конструктор, которые зануляют указатель, стоящий справо:

\begin{minted}[
linenos,
frame=lines,
framesep=2mm]
{c++}
unique_ptr& operator=(unique_ptr&& other) noexcept
{
    reset(other.release()); 
    return *this;
}

unique_ptr(unique_ptr&& other) noexcept
    : ptr(other.release())
{}
\end{minted}

\subsection{Владение}
Ответственность за удаление объекта называется владением. Например, \mintinline{c++}{std::unique_ptr} ответственен за удаление объекта на который он ссылается, соответственно говорят, что \mintinline{c++}{std::unique_ptr} владеет объектом, на который он ссылается. Про функцию \mintinline{c++}{reset(p)} говорят, что она передает владение объектом \mintinline{c++}{std::unique_ptr}'у, а функция \mintinline{c++}{release()}, наоборот, забирает владение объектом у \mintinline{c++}{std::unique_ptr}'а.

Термин владение применяется не только к умным указателям, например можно сказать, что std::vector владеет памятью выделенной под свои элементы (обязан её освободить), а также владеет своими элементами (обязан вызывать им деструктор).

В некоторых случаях объект может иметь несколько владельцев. Это называется разделяемым владением и работает следующим образом: пока существует хотя бы один владелец объект продолжает жить, когда пропадает последний владелец --- объект удаляется. Для умных указателей существует два способа реализации разделяемого владения: подсчет ссылок и провязка всех владельцев в двусвязный список. Оба подхода имеют свои преимущества и недостатки. Подсчет ссылок применяется во включенном в стандартную библиотеку указателе \mintinline{c++}{std::shared_ptr}. Указатель использующий провязку владельцев в двусвязный список в стандартной библиотеке отсутствует, но часто называется \mintinline{c++}{linked_ptr}.

\subsection{\mintinline{c++}{std::shared_ptr}}
\mintinline{c++}{std::shared_ptr} - это умный указатель, с разделяемым владением объектов через его указатель. Несколько указателей могут владеть одним объектом. Объект будет уничтожен, когда последний \mintinline{c++}{shared_ptr} будет уничтожен или сброшен.

\textcolor{red}{NB}) \mintinline{c++}{shared_ptr} может не указывать ни на какой объект.

\subsubsection{Наивная реализация.}

\begin{minted}[linenos, frame=lines, framesep=2mm, tabsize = 4, breaklines]{c++}
template <typename T, typename D>
struct shared_ptr {
    T* obj; // указатель на объект управления
    size_t* counter; // счетчик ссылок, обнуление которого влечет удаление объекта
    D *deleter; // функция, которая удаляет объект
    shared_ptr(*T, D const& deleter);
}

\end{minted}



\textcolor{red}{NB}) \textbf{Зачем делитер, если можно всегда вызвать деструктор объекта?} Ответ: Ингда мы хотим реализовать следующую схему пользования объектом: 1) мы берем объект из ресурса. 2) пользуемся им. 3) Потом возвращаем его от куда взяли, когда он стал нам не нужен.

\subsubsection{Оптимизация по памяти.}
На самом деле все немного сложнее: нам может понядобиться хранить в нашем smart\_ptr аллокатор, счетчик weak\_ptr (об этом позднее), указалеть на объект управления. Поэтому имеет смысл хранить все это в одном объекте, для оптимизации выделения памяти под smart\_ptr. С учетом выше указанного:

\begin{minted}[linenos, frame=lines, framesep=2mm, tabsize = 4, breaklines]{c++}
template <typename T, typename D>
struct shared_ptr {
    T* obj; // указатель на объект управления
    struct control_block {
        size_t* counter; // счетчик ссылок, обнуление которого влечет удаление объекта
        D *deleter; // функция, которая удаляет объект
    }* con_bl;
    shared_ptr(*T, D const& deleter);
};

\end{minted}


\textcolor{red}{NB}) если создавать \mintinline{c++}{std::shared_ptr} с помощью конструктора, то память будет распределяться как выше указанно, но если создать \mintinline{c++}{std::shared_ptr} с помошью \mintinline{c++}{std::make_shared(new T)}\footnote{создает объект и оборачивает его в \mintinline{c++}{std::shared_ptr}}, то память выделится только один раз, и объект будет лежать рядом с блоком управления. Поэтому лучше использовать последний способ где это возможно, чтобы снизить оверхед от "умности"\ указателя.

\subsubsection{Специальный конструтор.}
Иногда мы хотим продлевать объекту жизнь, если что-то ссылается на его поля. Для этого существует специальный конструтор:

\begin{minted}[linenos, frame=lines, framesep=2mm, tabsize = 4, breaklines]{c++}
struct car {
    Wheel wheel[4];
}
// У нас есть машина с колесами. Мы хотим, чтобы пока мы умеет живой указатель на колесо, машина тоже не удалялась вместе с колесами.
/* ... */
shared_ptr<car> p;
shared_ptr<wheel> q(&p, wheel[2]); // мы переиспользуем p.counter для q.

\end{minted}
Для этого наверно, будет полезно иметь ссылку на объект в блоке управления.\footnote{Мнение автора.}


\textcolor{red}{NB}) Еще одно применение этой фичи: мы хотим хранить дерево по \mintinline{c++}{std::shared_ptr}, чтобы когда мы удалили корень, дерево само удалиться. Тогда с помощью этого, конструктора можно гарантировать, что если есть указатель на вершины дерева, то можно гарантировать, что по нему можно пробежаться.

\subsubsection{Заметки по использованию.}
\begin{enumerate}
    \item
    Возможно глупый пример, но все же.
    \begin{minted}[
    linenos,
    frame=lines,
    framesep=2mm,
    tabsize = 4,
    breaklines]
    {c++}

    int *silly_ptr = new int(5);
    shared_ptr<int> a1(silly_ptr);
    shared_ptr<int> b1(silly_ptr);
    // b1.counter = a1.counter = 1;
    // разные управляющие блоки
    shared_ptr<int> a2(silly_ptr);
    shared_ptr<int> b2(a2);
    // b2.counter = a2.counter = 2;
    // одинаковые

    \end{minted}


\end{enumerate}
\subsubsection{Weak\_ptr}
\mintinline{c++}{std::weak_ptr} - умный указатель, который моделирует временное владение объектом.


Ситуация: мы хотим хешировать картинки, которые хранять по \mintinline{c++}{std::shared_ptr}. То есть нам нужен мэп: name\_image $\to$ \mintinline{c++}{std::shared_ptr}. Но есть проблема, что если мы будем хранить в мэпе \mintinline{c++}{std::shared_ptr}, то они будут считать владельцами этих картинок, и они никогда не будут удаляться (если только их не удалить руками). В этой ситуации может помочь \mintinline{c++}{std::weak_ptr}.


Дело в том, что weak\_ptr содержит "слабую"\ ссылку на объект, управляемый указателем \mintinline{c++}{std::shared_ptr}. Для этого и нужен счетчик "слабых" сслылок в блоке управления \mintinline{c++}{std::shared_ptr}. Теперь блок управления считает еще и слабые сслыки и когда удаляется управляемый объект, блок управления остается жить, пока на него ссылается хотя бы один weak\_ptr, чтобы сообщать им о существовании объекта. Например с помощью метода \mintinline[breaklines]{c++}{std::weak_ptr::expired() // проверяет удален ли объект.}


То есть все, что должен хранить внутри себя weak\_ptr - это указатель блок управления.


Теперь вернемся к нашей задаче: мы сделаем мэп: name\_image $\to$ weak\_ptr.
\begin{minted}[linenos, frame=lines, framesep=2mm, tabsize = 4, breaklines]{c++}
map<string, weak_ptr<image>> cache;
shared_ptr<image> load_image(string const& name) {
    auto i = cache.find(name);
    if (i != cache.end()) {
        shared_ptr<image> temp = i->second.lock(); // создаем shared_ptr, который управляет объектом, которым управляет i. Если объект удален temp будет пустой.
        if (temp) {// проверка на то, что temp не пуст
            return temp;
        }
    }
    // если картинки нет в кэше загружаем ее.
    shared_ptr<image> temp = real_load(name);
    cache[name] = temp; // переобразование weak_ptr(shared_ptr);
    return temp;
}

\end{minted}


\textcolor{red}{NB}) Преобразования между \mintinline{c++}{std::shared_ptr} и weak\_ptr - это нормально. Только с его помощью можно обратиться к объекту weak\_ptr, не определяя новых указателей.


\textcolor{red}{NB}) С помощью weak\_ptr можно решать циклические зависимости \mintinline{c++}{std::shared_ptr}. Суть проблемы в том, что объекты не могут удалиться, так как ссылаются друг на друга.

\subsection{Сast\_pointer}
Что выполнять безоспасное приведение умных указателей можно искользовать слудующий синтаксис:
\begin{minted}[linenos, frame=lines, framesep=2mm, tabsize = 4, breaklines]{c++}
struct A {/*...*/};
struct B: A {/*...*/};

shared_ptr<A> foo;
shared_ptr<B> bar;
/*...*/
bar = make_shared<B>();

foo = dynamic_pointer_cast<A>(bar);
// иначе пришлось бы писать так:
foo = shared_ptr<A>(dynamic_cast<A>(bar.get()));

\end{minted}

\subsection{\mintinline{c++}{linked_ptr}}

\mintinline{c++}{linked_ptr} --- умный указатель с разделяемым владением, реализованный с помощью двусвязного списка.

\mintinline{c++}{linked_ptr} хранит в себе указатель на объект и два указателя на соседние \mintinline{c++}{linked_ptr}'ы в двусвязном списке. Для каждого двусвязного списка образованного из \mintinline{c++}{linked_ptr}'ов верно, что все указатели в нём владеют одним общим объектом.

\begin{minted}[
linenos,
frame=lines,
framesep=2mm]
{c++}
template <class T> struct linked_ptr
{
    ...
private:
    T* ptr;
    mutable linked_ptr* prev; 
    mutable linked_ptr* next; 
};
\end{minted}
Члены prev и next помечены как \mintinline{c++}{mutable}, потому что в случае если мы имеем константный \mintinline{c++}{const linked_ptr& p1}, который передается в функцию, и неконстантный \mintinline{c++}{linked_ptr& p2}, который является его соседом в двусвязном списке, то при изменении p1 придется поменять prev и next у p2.

Так, например, может выглядеть фрагмент двусвязного списка, образованного \mintinline{c++}{linked_ptr}'ами ptr1, ptr2, ptr3, указывающими на общий объект object :

\begin{minipage}[h]{0.49\linewidth}
\centering
\begin{asy}
real arrow_len = 5;
real box_h = arrow_len / 3.0;
real box_w = box_h;
real box_count = 3;
real object_box_w = 2 * box_w;

real get_x0(int i) {
    return arrow_len * i + box_w * (i - 1);
}

real get_x1(int i) {
    return get_x0(i) + box_w;
}

real obj_x0 = get_x0(2) - box_w / 2.0;
real obj_x1 = obj_x0 + object_box_w;
real obj_y0 = 0;
real obj_y1 = box_h;

real eps = arrow_len / 10.0;

real obj_label_x = (obj_x0 + obj_x1) / 2.0;
real obj_label_y = (obj_y0 + obj_y1) / 2.0;

real box_y0 = obj_y1 + arrow_len;
real box_y1 = box_y0 + box_h;

real box_label_x_offset = box_w / 2.0;
real box_label_y_offset = box_h / 2.0;

real arrow_left_offset = box_h * 2.0 / 3.0;
real arrow_right_offset = box_h * 1.0 / 3.0;
real arrow_left_y = box_y0 + arrow_left_offset;
real arrow_right_y = box_y0 + arrow_right_offset;
real arrow_down_offset = box_w * 0.5;

real label_offset = box_h / 6.0;
size(15cm,0);

void draw_arrow(path p, real relative_pos, string text, align al) {
    draw(
        L=Label(
            scale(0.5)*("" + text + ""),
            position=Relative(relative_pos),
            gray,
            align = al
        ),
        p,
        arrow=Arrow
    );
}

void draw_strait_arrows(real x0, real x1, real len = arrow_len) {
    real rel_pos = 0.75 + 0.22 * (1 - arrow_len / len);
    draw_arrow((x0 - len, arrow_left_y) -- (x0, arrow_left_y), rel_pos, "next", (E + N / 2.0));
    draw_arrow((x0, arrow_right_y) -- (x0 - len, arrow_right_y), rel_pos, "prev", (W - N / 2.0));
}

void draw_curved_arrow(pair p0, pair p3, string text, bool reversed = false) {
    real delta = arrow_len * (1.0 / 3.0);
    pair p1 = (xpart(p0) + delta, ypart(p0));
    pair p2 = (xpart(p3) - delta, ypart(p3));
    pair add = (arrow_len / 10.0, 0);
    if (reversed) {
        p1 -= add;
        p2 -= add - (0, box_h / 10.0);
    } else {
        p1 += add - (0, box_h / 10.0);
        p2 += add;
    }
    p2 += 0.3 * (p1 - p2);
    p1 -= 0.3 * (p1 - p2);
    pair p1_5 = (p1 + p2) / 2.0;

    //draw(circle(p1, 0.1), red);
    //draw(circle(p1_5, 0.1), magenta);
    //draw(circle(p2, 0.1), blue);

    path p = p0{right}..{down}p1{down}..{left}p1_5{left}..{down}p2{down}..{right}p3;
    draw_arrow(reversed ? reverse(p) : p, 0.938, text, reversed ? N : -N);

    pair q0 = p3 + box_w, q1 = p2 + (box_w + 2 * (xpart(p3) - xpart(p2)), 0), q3 = p0 + arrow_len, q2 = p1 + (arrow_len - 2 * (xpart(p1) - xpart(p0)), 0);
    pair q1_5 = (q1 + q2) / 2.0;

    //draw(circle(q1, 0.1), red);
    //draw(circle(q1_5, 0.1), magenta);
    //draw(circle(q2, 0.1), blue);

    p = q0{right}..{up}q1{up}..{left}q1_5{left}..{up}q2{up}..{right}q3;
    draw_arrow(reversed ? reverse(p) : p, 0.938, text, reversed ? N : -N);
}

void draw_arrow_to_obj(real x0, real y0) {
    pair p0 = (x0, y0);
    real obj_mid = (obj_x0 + obj_x1) / 2.0;
    pair p1 = (obj_mid + (x0 - obj_mid) / 1.5, obj_y1 + arrow_len / 2.0 - abs(x0 - obj_mid) / 7.0);
    pair p2 = (obj_mid+ (x0 - obj_mid) / 50.0, obj_y1);
    if (x0 == obj_mid) {
        draw(
            L=Label(
                scale(0.5)*"ptr",
                position=Relative(0.7),
                gray,
                align=-N + W * 0.7
            ),
            p0 -- p2,
            arrow=Arrow
        );
    } else if (x0 < obj_mid) 
        draw(
            L=Label(
                scale(0.5)*"ptr",
                position=Relative(0.6),
                gray,
                align=-N
            ),
            p0{down}..{right}p1{right}..{down}p2,
            arrow=Arrow
        );
    else
        draw(
            L=Label(
                scale(0.5)*"ptr",
                position=Relative(0.5),
                gray,
                align=-N
            ),
            p0{down}..{left}p1{left}..{down}p2,
            arrow=Arrow
        );
}

void draw_box(real x0, real x1, int i, real y0 = box_y0, real y1 = box_y1, pen color=royalblue) {
    draw_arrow_to_obj(x0 + arrow_down_offset, y0);
    path ptr = box((x0, y0), (x1, y1));
    fill(ptr, color);
    draw(ptr, black);
    label("ptr" + string(i) + "", (x0 + box_label_x_offset, y0 + box_label_y_offset), white);
}

void draw_last_arrows() {
    int last_index = (int)box_count + 1;
    real last_x0 = get_x0(last_index) - arrow_len / 2;
    real last_x1 = get_x1(last_index) - arrow_len / 2;
    draw_strait_arrows(last_x0, last_x1, arrow_len / 2);
}


path obj = box((obj_x0,obj_y0), (obj_x1,obj_y1));
fill(obj, deepgreen);
draw(obj, black);
label("object", (obj_label_x, obj_label_y), white);


for (int i = 1; i <= box_count; ++i) {
    real x0 = get_x0(i), x1 = get_x1(i);
    draw_strait_arrows(x0, x1, i == 1 ? arrow_len / 2 : arrow_len);
    draw_box(x0, x1, i);
}
draw_last_arrows();
\end{asy}
\label{fig:prob1_6_2}
\end{minipage}

При копировании \mintinline{c++}{linked_ptr<int> ptr4 = ptr2;} новый ptr4 вставляется в тот же список, где был исходный ptr2. После выполнения этой операции двусвязный список указателей будет выглядеть следующим образом:

\begin{minipage}[h]{0.49\linewidth}
\centering
\begin{asy}
real arrow_len = 5;
real box_h = arrow_len / 3.0;
real box_w = box_h;
real box_count = 3;
real object_box_w = 2 * box_w;

real get_x0(int i) {
    return arrow_len * i + box_w * (i - 1);
}

real get_x1(int i) {
    return get_x0(i) + box_w;
}

real obj_x0 = get_x0(2) - box_w / 2.0;
real obj_x1 = obj_x0 + object_box_w;
real obj_y0 = 0;
real obj_y1 = box_h;

real eps = arrow_len / 10.0;

real obj_label_x = (obj_x0 + obj_x1) / 2.0;
real obj_label_y = (obj_y0 + obj_y1) / 2.0;

real box_y0 = obj_y1 + arrow_len;
real box_y1 = box_y0 + box_h;

real box_label_x_offset = box_w / 2.0;
real box_label_y_offset = box_h / 2.0;

real arrow_left_offset = box_h * 2.0 / 3.0;
real arrow_right_offset = box_h * 1.0 / 3.0;
real arrow_left_y = box_y0 + arrow_left_offset;
real arrow_right_y = box_y0 + arrow_right_offset;
real arrow_down_offset = box_w * 0.5;

real label_offset = box_h / 6.0;
size(15cm,0);

void draw_arrow(path p, real relative_pos, string text, align al) {
    draw(
        L=Label(
            scale(0.5)*("" + text + ""),
            position=Relative(relative_pos),
            gray,
            align = al
        ),
        p,
        arrow=Arrow
    );
}

void draw_strait_arrows(real x0, real x1, real len = arrow_len) {
    real rel_pos = 0.75 + 0.22 * (1 - arrow_len / len);
    draw_arrow((x0 - len, arrow_left_y) -- (x0, arrow_left_y), rel_pos, "next", (E + N / 2.0));
    draw_arrow((x0, arrow_right_y) -- (x0 - len, arrow_right_y), rel_pos, "prev", (W - N / 2.0));
}

void draw_curved_arrow(pair p0, pair p3, string text, bool reversed = false) {
    real delta = arrow_len * (1.0 / 3.0);
    pair p1 = (xpart(p0) + delta, ypart(p0));
    pair p2 = (xpart(p3) - delta, ypart(p3));
    pair add = (arrow_len / 10.0, 0);
    if (reversed) {
        p1 -= add;
        p2 -= add - (0, box_h / 10.0);
    } else {
        p1 += add - (0, box_h / 10.0);
        p2 += add;
    }
    p2 += 0.3 * (p1 - p2);
    p1 -= 0.3 * (p1 - p2);
    pair p1_5 = (p1 + p2) / 2.0;

    //draw(circle(p1, 0.1), red);
    //draw(circle(p1_5, 0.1), magenta);
    //draw(circle(p2, 0.1), blue);

    path p = p0{right}..{down}p1{down}..{left}p1_5{left}..{down}p2{down}..{right}p3;
    draw_arrow(reversed ? reverse(p) : p, 0.938, text, reversed ? N : -N);

    pair q0 = p3 + box_w, q1 = p2 + (box_w + 2 * (xpart(p3) - xpart(p2)), 0), q3 = p0 + arrow_len, q2 = p1 + (arrow_len - 2 * (xpart(p1) - xpart(p0)), 0);
    pair q1_5 = (q1 + q2) / 2.0;

    //draw(circle(q1, 0.1), red);
    //draw(circle(q1_5, 0.1), magenta);
    //draw(circle(q2, 0.1), blue);

    p = q0{right}..{up}q1{up}..{left}q1_5{left}..{up}q2{up}..{right}q3;
    draw_arrow(reversed ? reverse(p) : p, 0.938, text, reversed ? N : -N);
}

void draw_arrow_to_obj(real x0, real y0) {
    pair p0 = (x0, y0);
    real obj_mid = (obj_x0 + obj_x1) / 2.0;
    pair p1 = (obj_mid + (x0 - obj_mid) / 1.5, obj_y1 + arrow_len / 2.0 - abs(x0 - obj_mid) / 7.0);
    pair p2 = (obj_mid+ (x0 - obj_mid) / 50.0, obj_y1);
    if (x0 == obj_mid) {
        draw(
            L=Label(
                scale(0.5)*"ptr",
                position=Relative(0.6),
                gray,
                align=-N + W * 0.7
            ),
            p0 -- p2,
            arrow=Arrow
        );
    } else if (x0 < obj_mid) 
        draw(
            L=Label(
                scale(0.5)*"ptr",
                position=Relative(0.5),
                gray,
                align=-N
            ),
            p0{down}..{right}p1{right}..{down}p2,
            arrow=Arrow
        );
    else
        draw(
            L=Label(
                scale(0.5)*"ptr",
                position=Relative(0.5),
                gray,
                align=-N
            ),
            p0{down}..{left}p1{left}..{down}p2,
            arrow=Arrow
        );
}

void draw_box(real x0, real x1, int i, real y0 = box_y0, real y1 = box_y1, pen color=royalblue) {
    draw_arrow_to_obj(x0 + arrow_down_offset, y0);
    path ptr = box((x0, y0), (x1, y1));
    fill(ptr, color);
    draw(ptr, black);
    label("ptr" + string(i) + "", (x0 + box_label_x_offset, y0 + box_label_y_offset), white);
}

void draw_last_arrows() {
    int last_index = (int)box_count + 1;
    real last_x0 = get_x0(last_index) - arrow_len / 2;
    real last_x1 = get_x1(last_index) - arrow_len / 2;
    draw_strait_arrows(last_x0, last_x1, arrow_len / 2);
}


path obj = box((obj_x0,obj_y0), (obj_x1,obj_y1));
fill(obj, deepgreen);
draw(obj, black);
label("object", (obj_label_x, obj_label_y), white);


for (int i = 1; i <= 2; ++i) {
    real x0 = get_x0(i), x1 = get_x1(i);
    draw_strait_arrows(x0, x1, i == 1 ? arrow_len / 2 : arrow_len);
    draw_box(x0, x1, i);
}
draw_last_arrows();

real x2_0 = get_x0(2), x2_1 = get_x1(2), x3_0 = get_x0(3), x3_1 = get_x1(3);
real x4_0 = x2_1 + (arrow_len - box_w) / 2.0;
real x4_1 = x4_0 + box_w;
real y4_0 = box_y0 - arrow_len / 2.0;
real y4_1 = y4_0 + box_h;

draw_box(x3_0, x3_1, 3);

draw_box(x4_0, x4_1, 4, y4_0, y4_1, brown);

draw_curved_arrow((x2_1, arrow_left_y), (x4_0, y4_0 + arrow_left_offset), "next", false);
draw_curved_arrow((x2_1, arrow_right_y), (x4_0, y4_0 + arrow_right_offset), "prev", true);

//draw_curved_arrow((x4_1, y4_0 + arrow_left_offset), (x3_0, arrow_left_y), "next", false);
//draw_curved_arrow((x4_1, y4_0 + arrow_right_offset), (x3_0, arrow_right_y), "prev", true);
\label{fig:prob1_6_2}
\end{asy}
\end{minipage}

\begin{minted}[
linenos,
frame=lines,
framesep=2mm]
{c++}
linked_ptr(linked_ptr const& other) noexcept
    : ptr(other.ptr)
{
    if (ptr == nullptr) {
        prev = next = nullptr;
        return;
    }
    prev = &other;
    next = other.next;
    prev->next = this;
    next->prev = this;
}
\end{minted}

При удалении \mintinline{c++}{linked_ptr} он удаляется из двусвязного списка. Если он был единственным элементом списка то он удаляет сам объект:
\begin{minted}[
linenos,
frame=lines,
framesep=2mm]
{c++}

~linked_ptr() noexcept 
{
    if (!ptr) return;
    if (prev == this)
        delete ptr;
    if (prev) prev->next = next;
    if (next) next->prev = prev;
}

\end{minted}

Move-конструктор у \mintinline{c++}{linked_ptr} может выглядеть следующим образом:

\begin{minted}[
linenos,
frame=lines,
framesep=2mm]
{c++}
linked_ptr(linked_ptr&& other) noexcept
    : ptr(other.ptr)
    , prev(other.prev)
    , next(other.next)
{
    other.prev = other.next = nullptr;
    other.ptr = nullptr;
    if (this->prev) 
        this->prev->next = this;
    if (this->next) 
        this->next->prev = this;
}
\end{minted}
Заметим, что в отличае от конструктора копирования нужно занулять члены other, потому что на нем будет вызван деструктор.

\mintinline{c++}{linked_ptr} реализует операции \mintinline{c++}{get()}, \mintinline{c++}{operator*} и \mintinline{c++}{operator->}, как и другие умные указатели:

\begin{minted}[
linenos,
frame=lines,
framesep=2mm]
{c++}
T* get() const noexcept { return ptr; } // разыменование указателя
T* operator->() const noexcept { return ptr; } // вызов мембера объекта, на который указывает linked_ptr
T& operator*() const { return *ptr; }
\end{minted}

\subsubsection{сравнение \mintinline{c++}{linked_ptr} и \mintinline{c++}{shared_ptr}}

\begin{enumerate}
\item \mintinline{c++}{std::shared_ptr} при передаче в конструктор указателя на объект аллоцирует еще 1 вспомогательный объект на хипе, а \mintinline{c++}{linked_ptr} - ничего. 
\item \mintinline{c++}{linked_ptr} имеет больший размер, чем \mintinline{c++}{shared_ptr} --- три указателя, по сравненею с двумя у \mintinline{c++}{shared_ptr}.
\item \mintinline{c++}{std::shared_ptr} в стандартной библиотеке является thread-safe. Он использует атомарные операции инкремента и декремента для счетчика ссылок. Thread-safe \mintinline{c++}{linked_ptr} не может быть так просто реализован и требует использования уже thread-safe структур данных.
\item \mintinline{c++}{linked_ptr} также имеет внутри больше присваиваний и сравнений на каждую операцию копирования и присваивания, в то время как \mintinline{c++}{std::shared_ptr} должен только инкрементировать счетчик.
\end{enumerate}

\subsection{\mintinline{c++}{intrusive_ptr}}

\mintinline{c++}{intrusive_ptr} - также умный указатель с разделяемым владением, реализованный в библиотеке Boost. Так же как и \mintinline{c++}{std::shared_ptr}, он использует подсчет ссылок, но в отличии от \mintinline{c++}{std::shared_ptr} в \mintinline{c++}{boost::intrusive_ptr<T>} счетчик ссылок хранится в самом объекте класса T, на который он ссылается.

Для использования \mintinline{c++}{intrusive_ptr<T>} требуется реализовать функции:

\begin{minted}[
linenos,
frame=lines,
framesep=2mm]
{c++}
int intrusive_ptr_add_ref(T* p);
int intrusive_ptr_release(T* p);
\end{minted}

Пример использования:

\begin{minted}[
linenos,
frame=lines,
framesep=2mm]
{c++}
#include <boost/intrusive_ptr.hpp>

class Test {
public:
    int ref_count;
};

void intrusive_ptr_add_ref(Test *p) {
    ++p->ref_count;
}

void intrusive_ptr_release(Test *p) {
    if (0 == --p->ref_count)
        delete p;
}

int main() {
    boost::intrusive_ptr<Test> p(new Test()); // аллоцированный объект, переданный в конструктор будет удален при выходе из main
}
\end{minted}

Замечание: \mintinline{c++}{boost::intrusive_ptr} эффективнее \mintinline{c++}{std::shared_ptr}, поскольку пользователь сам может встроить эффективный счетчик ссылок для своего объекта. Причем контракт функций \mintinline{c++}{intrusive_ptr_add_ref} и \mintinline{c++}{intrusive_ptr_release} такой, что они не накладывают строгие ограничения на счетчик ссылок, кроме того, что \mintinline{c++}{intrusive_ptr_release} должен удалить объект, когда удаляется последний его владелец. Поэтому если планируется использование разделяемого smart-pointer’а для класса, реализуемого самим пользователем, то лучше юзать \mintinline{c++}{intrusive_ptr}, а в случае если приходится иметь дело с библиотечным классом, то \mintinline{c++}{std::shared_ptr}.

\subsection{make\_shared / make\_unique}

Как было сказано ранее, помимо стандартных конструкторов от указателя стандарт предлагает фабрики для создания объектов, определенные следующим образом (рассмотрим на примере \mintinline{c++}{std::shared_ptr}): 

\begin{minted}[
linenos,
frame=lines,
framesep=2mm]
{c++}
template<typename T, typename... Args>
shared_ptr<T> make_shared(Args&&... args); // создает std::shared_ptr<T>, указывающий на объект T(args...)
\end{minted}
Исторически введение данного метода должно было решить проблему производительности \mintinline{c++}{std::shared_ptr}, аллоцируя внутри только один объект, содержащий одновременно и счетчик ссылок и объект типа T. Причем при вызове \mintinline{c++}{std::make_shared(args...)} на новом объекте типа T сразу вызовется конструктор с переданными аргументами. Таким образом такой конструктор-фабрика экономит аллокации памяти (вместо двух аллокаций: \mintinline{c++}{new T(args...)} и \mintinline{c++}{new counter()} происходит одна аллокация данных для пары (\mintinline{c++}{object}, \mintinline{c++}{counter}).

Также введение такой функции в стандарт позволило решить проблему с гарантиями безопасности. Представим следующую ситуацию :
\begin{minted}[
linenos,
frame=lines,
framesep=2mm]
{c++}
f(std::shared_ptr<Class>(new Class(“class”)), g());
\end{minted}

При этом \mintinline{c++}{g()} бросает исключение. Т.к. по стандарту нет гарантии порядка вычисления аргументов \mintinline{c++}{f}, то может произойти следующее : сначала создадим \mintinline{c++}{new Class()}, выделим память, потом - вызовем \mintinline{c++}{g()}, которое бросит исключение, после чего конструктор \mintinline{c++}{shared_ptr} уже не вызовется, т.к. бросился exception. А т.к. мы нигде не пишем слово \mintinline{c++}{delete}, то произойдет memory leak.

Теперь рассмотрим случай с \mintinline{c++}{make_shared} : 

\begin{minted}[
linenos,
frame=lines,
framesep=2mm]
{c++}
f(std::make_shared<Class>(“class”), g());
\end{minted}

Если сначала вызовется \mintinline{c++}{g()}, то до конструктора \mintinline{c++}{Class} дело вообще не дойдет и упомянутой проблемы не возникнет. С другой стороны, если сначала вызовется \mintinline{c++}{make_shared}, то объект будет уже обернут в smart pointer и после исключения в \mintinline{c++}{g()} он будет автоматически удален!!!

Начиная с c++14 аналогичная функция-фабрика \mintinline{c++}{std::make_unique} была добавлена и для \mintinline{c++}{std::unique_ptr}. Это позволило не использовать в коде не только \mintinline{c++}{delete}, но и голый \mintinline{c++}{new} (no naked new) для обоих видов умных указателей, включенных в стандартную библиотеку.

\subsection{Smart pointers pointing on this}

Особый случай, когда нельзя просто так взять и использовать умные указатели бездумно --- это указатели на this, которые возвращает метод класса. Рассмотрим код :

\begin{minted}[
linenos,
frame=lines,
framesep=2mm]
{c++}

class SomeClass {
	int data = 5;
	std::shared_ptr f() {
		return std::shared_ptr(this);
	}
	void foo() {}
}

SomeClass obj;

if (true) { // some scope
	auto ptr = ob.f();
} // exit scope (*)
obj.foo();

\end{minted}

В этом примере при выхода из скоупа (*) т.к. умный указатель на obj создан всего 1, он вызовет деструктор \mintinline{c++}{obj}. Но при этом после этого мы уже не сможем его использовать (\mintinline{c++}{obj.foo()} уже некорректно). Поэтому для таких целей в stdlib есть интерфейс \mintinline{c++}{std::enable_shared_from_this}, оторый позволяет держать сильную ссылку на себя внутри самого объекта класса, который от него унаследован. В нашем случае :


\begin{minted}[
linenos,
frame=lines,
framesep=2mm]
{c++}

class SomeClass: std::enable_shared_from_this<SomeClass> {
	int data = 5;
	std::shared_ptr f() {
		return std::shared_ptr(this);
	}
	void foo() {}
}


SomeClass obj;


if (true) { 
	auto ptr = ob.f();
} 
obj.foo();

\end{minted}

Такой код будет уже корректный.

\subsection{smart pointers и наследование}

В стандартной библиотеке умные указатели сделаны так, что если присваивания объекта типа \mintinline{c++}{A} может быть произведено в объект типа \mintinline{c++}{B}, то присваивания объекта типа \mintinline{c++}{smart_pointer<A>} может быть произведено в объект типа \mintinline{c++}{smart_pointer<B>}.
В том числе:
 
\begin{minted}[
linenos,
frame=lines,
framesep=2mm]
{c++}

shared_ptr<Derived> dp1(new Derived); 
shared_ptr<Base> bp1 = dp1; 
shared_ptr<Base> bp2(dp1); 
shared_ptr<Base> bp3(new Derived);

\end{minted}
Это корректный код. Но стоит отметить, что в этом случае т.к. указатель внутри будет хранится на \mintinline{c++}{Derived} объект, то деструктор будет вызван ровно его. Поэтому если он не объявлен как виртуальный, то это UB.
Для явного каста указателей в стандартной библиотеке есть специальные методы:
\begin{minted}[
linenos,
frame=lines,
framesep=2mm]
{c++}

derived_ptr = static_pointer_cast<Derived>(base_ptr);

\end{minted}
Этот стейтмент валиден т.и.т.т, когда \mintinline{c++}{static_cast<Derived *>(base_ptr.get())} - валидно, т.к. Ровно на его основе метод \mintinline{c++}{static_pointer_cast} реализован. Его основное преимущество - отсутствие необходимости писать \mintinline{c++}{.get()} вручную и тем самым необходимости работать с классическими голыми указателями.

\subsection{доп. функции}
\begin{enumerate}
\item Умные указатели умеют каститься к \mintinline{c++}{bool} (\mintinline{c++}{true} т.и т.т., когда указываемый объект не \mintinline{c++}{nullptr})
\item  Также для них определены операции сравнения \mintinline{c++}{==}, \mintinline{c++}{!=}, \mintinline{c++}{<}, которые сравнивают голые указатели внутри.
\end{enumerate}

\subsection{Performance test}

Также уместно представить сравнение производительностей разных видов указателей:

\begin{minipage}[h]{0.49\linewidth}
\centering
\begin{asy}
real linked = 221;
real shared = 572;
real intrusive = 214;
real unique = 239;
real linked_disp = 0;
real shared_disp = 2;
real intrusive_disp = 8;
real unique_disp = 6;real bench_width = 312.0;
real max_height = 572;

real arrow_offset = bench_width / 4.0 * 1.1;
real point_offset = arrow_offset * 2 / 3;
real linked_offset = arrow_offset + bench_width;
real shared_offset = arrow_offset + 2 * bench_width;
real intrusive_offset = arrow_offset + 3 * bench_width;
real unique_offset = arrow_offset + 4 * bench_width;
real label_offset = (shared_offset - linked_offset) / 2.0;

size(20cm,0);
draw((arrow_offset / 2, 0) -- (arrow_offset / 2, max_height),arrow=Arrow);

label(
    scale(0.5) * ("" + string(linked) + ""),
    (point_offset, linked),
    black
);
label(
    scale(0.5) * ("" + string(shared) + ""),
    (point_offset, shared),
    black
);
label(
    scale(0.5) * ("" + string(intrusive) + ""),
    (point_offset, intrusive),
    black
);
label(
    scale(0.5) * ("" + string(unique) + ""),
    (point_offset, unique),
    black
);
path lin = box((arrow_offset,0), (linked_offset,linked));
fill(lin, deepgreen);

path sha = box((linked_offset,0), (shared_offset,shared));
fill(sha, royalblue);

path intr = box((shared_offset,0), (intrusive_offset,intrusive));
fill(intr, magenta);

path uniq = box((intrusive_offset,0), (unique_offset,unique));
fill(uniq, heavyred);

real disp_eps = bench_width / 20.0;
draw(
    (arrow_offset, linked) 
    -- (arrow_offset, linked + linked_disp) 
    -- (arrow_offset + disp_eps, linked + linked_disp) 
    -- (arrow_offset - disp_eps, linked + linked_disp) 
    -- (arrow_offset, linked + linked_disp)
    -- (arrow_offset, linked - linked_disp)
    -- (arrow_offset + disp_eps, linked - linked_disp) 
    -- (arrow_offset - disp_eps, linked - linked_disp) 
    -- (arrow_offset, linked - linked_disp)
, gray);
draw(
    (linked_offset, shared) 
    -- (linked_offset, shared + shared_disp) 
    -- (linked_offset + disp_eps, shared + shared_disp) 
    -- (linked_offset - disp_eps, shared + shared_disp) 
    -- (linked_offset, shared + shared_disp)
    -- (linked_offset, shared - shared_disp)
    -- (linked_offset + disp_eps, shared - shared_disp) 
    -- (linked_offset - disp_eps, shared - shared_disp) 
    -- (linked_offset, shared - shared_disp)
, gray);
draw(
    (shared_offset, intrusive) 
    -- (shared_offset, intrusive + intrusive_disp) 
    -- (shared_offset + disp_eps, intrusive + intrusive_disp) 
    -- (shared_offset - disp_eps, intrusive + intrusive_disp) 
    -- (shared_offset, intrusive + intrusive_disp)
    -- (shared_offset, intrusive - intrusive_disp)
    -- (shared_offset + disp_eps, intrusive - intrusive_disp) 
    -- (shared_offset - disp_eps, intrusive - intrusive_disp) 
    -- (shared_offset, intrusive - intrusive_disp)
, gray);
draw(
    (intrusive_offset, unique) 
    -- (intrusive_offset, unique + unique_disp) 
    -- (intrusive_offset + disp_eps, unique + unique_disp) 
    -- (intrusive_offset - disp_eps, unique + unique_disp) 
    -- (intrusive_offset, unique + unique_disp)
    -- (intrusive_offset, unique - unique_disp)
    -- (intrusive_offset + disp_eps, unique - unique_disp) 
    -- (intrusive_offset - disp_eps, unique - unique_disp) 
    -- (intrusive_offset, unique - unique_disp)
, gray);

real label_delta = 0.05 * max(max(linked, shared), max(intrusive, unique));
label("linked\underline{\hspace{0.3cm}}ptr", (linked_offset - label_offset, linked + label_delta), black);
label("std::shared\underline{\hspace{0.3cm}}ptr", (shared_offset - label_offset, shared + label_delta), black);
label("boost::intruisive\underline{\hspace{0.3cm}}ptr", (intrusive_offset - label_offset, intrusive + label_delta), black);
label("std::unique\underline{\hspace{0.3cm}}ptr", (unique_offset - label_offset, unique + label_delta), black);
label("allocation/deallocation benchmark : ", (bench_width * 2.0, max_height * 1.1));
real linked = 178;
real shared = 357;
real intrusive = 195;
real unique = 178;
real linked_disp = 4;
real shared_disp = 9;
real intrusive_disp = 3;
real unique_disp = 4;path lin = box((arrow_offset,0), (linked_offset,linked));
fill(lin, black);

path sha = box((linked_offset,0), (shared_offset,shared));
fill(sha, black);

path intr = box((shared_offset,0), (intrusive_offset,intrusive));
fill(intr, black);

path uniq = box((intrusive_offset,0), (unique_offset,unique));
fill(uniq, black);

real disp_eps = bench_width / 20.0;
draw(
    (arrow_offset, linked) 
    -- (arrow_offset, linked + linked_disp) 
    -- (arrow_offset + disp_eps, linked + linked_disp) 
    -- (arrow_offset - disp_eps, linked + linked_disp) 
    -- (arrow_offset, linked + linked_disp)
    -- (arrow_offset, linked - linked_disp)
    -- (arrow_offset + disp_eps, linked - linked_disp) 
    -- (arrow_offset - disp_eps, linked - linked_disp) 
    -- (arrow_offset, linked - linked_disp)
, gray);
draw(
    (linked_offset, shared) 
    -- (linked_offset, shared + shared_disp) 
    -- (linked_offset + disp_eps, shared + shared_disp) 
    -- (linked_offset - disp_eps, shared + shared_disp) 
    -- (linked_offset, shared + shared_disp)
    -- (linked_offset, shared - shared_disp)
    -- (linked_offset + disp_eps, shared - shared_disp) 
    -- (linked_offset - disp_eps, shared - shared_disp) 
    -- (linked_offset, shared - shared_disp)
, gray);
draw(
    (shared_offset, intrusive) 
    -- (shared_offset, intrusive + intrusive_disp) 
    -- (shared_offset + disp_eps, intrusive + intrusive_disp) 
    -- (shared_offset - disp_eps, intrusive + intrusive_disp) 
    -- (shared_offset, intrusive + intrusive_disp)
    -- (shared_offset, intrusive - intrusive_disp)
    -- (shared_offset + disp_eps, intrusive - intrusive_disp) 
    -- (shared_offset - disp_eps, intrusive - intrusive_disp) 
    -- (shared_offset, intrusive - intrusive_disp)
, gray);
draw(
    (intrusive_offset, unique) 
    -- (intrusive_offset, unique + unique_disp) 
    -- (intrusive_offset + disp_eps, unique + unique_disp) 
    -- (intrusive_offset - disp_eps, unique + unique_disp) 
    -- (intrusive_offset, unique + unique_disp)
    -- (intrusive_offset, unique - unique_disp)
    -- (intrusive_offset + disp_eps, unique - unique_disp) 
    -- (intrusive_offset - disp_eps, unique - unique_disp) 
    -- (intrusive_offset, unique - unique_disp)
, gray);

real label_heigth=bench_width / 4.0;
label("heap allocation", (linked_offset - label_offset, label_heigth), white);
label("heap allocation", (shared_offset - label_offset, label_heigth), white);
label("heap allocation", (intrusive_offset - label_offset, label_heigth), white);
label("heap allocation", (unique_offset - label_offset, label_heigth), white);
\end{asy}
\end{minipage}

\begin{minipage}[h]{0.49\linewidth}
\begin{asy}
real linked = 12;
real shared = 48;
real intrusive = 10;
real unique = 0;
real linked_disp = 0;
real shared_disp = 0;
real intrusive_disp = 0;
real unique_disp = 0;real bench_width = 18.0;
real max_height = 48;

real arrow_offset = bench_width / 4.0 * 1.1;
real point_offset = arrow_offset * 2 / 3;
real linked_offset = arrow_offset + bench_width;
real shared_offset = arrow_offset + 2 * bench_width;
real intrusive_offset = arrow_offset + 3 * bench_width;
real unique_offset = arrow_offset + 4 * bench_width;
real label_offset = (shared_offset - linked_offset) / 2.0;

size(20cm,0);
draw((arrow_offset / 2, 0) -- (arrow_offset / 2, max_height),arrow=Arrow);

label(
    scale(0.5) * ("" + string(linked) + ""),
    (point_offset, linked),
    black
);
label(
    scale(0.5) * ("" + string(shared) + ""),
    (point_offset, shared),
    black
);
label(
    scale(0.5) * ("" + string(intrusive) + ""),
    (point_offset, intrusive),
    black
);
label(
    scale(0.5) * ("" + string(unique) + ""),
    (point_offset, unique),
    black
);
path lin = box((arrow_offset,0), (linked_offset,linked));
fill(lin, deepgreen);

path sha = box((linked_offset,0), (shared_offset,shared));
fill(sha, royalblue);

path intr = box((shared_offset,0), (intrusive_offset,intrusive));
fill(intr, magenta);

path uniq = box((intrusive_offset,0), (unique_offset,unique));
fill(uniq, heavyred);

real disp_eps = bench_width / 20.0;
draw(
    (arrow_offset, linked) 
    -- (arrow_offset, linked + linked_disp) 
    -- (arrow_offset + disp_eps, linked + linked_disp) 
    -- (arrow_offset - disp_eps, linked + linked_disp) 
    -- (arrow_offset, linked + linked_disp)
    -- (arrow_offset, linked - linked_disp)
    -- (arrow_offset + disp_eps, linked - linked_disp) 
    -- (arrow_offset - disp_eps, linked - linked_disp) 
    -- (arrow_offset, linked - linked_disp)
, gray);
draw(
    (linked_offset, shared) 
    -- (linked_offset, shared + shared_disp) 
    -- (linked_offset + disp_eps, shared + shared_disp) 
    -- (linked_offset - disp_eps, shared + shared_disp) 
    -- (linked_offset, shared + shared_disp)
    -- (linked_offset, shared - shared_disp)
    -- (linked_offset + disp_eps, shared - shared_disp) 
    -- (linked_offset - disp_eps, shared - shared_disp) 
    -- (linked_offset, shared - shared_disp)
, gray);
draw(
    (shared_offset, intrusive) 
    -- (shared_offset, intrusive + intrusive_disp) 
    -- (shared_offset + disp_eps, intrusive + intrusive_disp) 
    -- (shared_offset - disp_eps, intrusive + intrusive_disp) 
    -- (shared_offset, intrusive + intrusive_disp)
    -- (shared_offset, intrusive - intrusive_disp)
    -- (shared_offset + disp_eps, intrusive - intrusive_disp) 
    -- (shared_offset - disp_eps, intrusive - intrusive_disp) 
    -- (shared_offset, intrusive - intrusive_disp)
, gray);
draw(
    (intrusive_offset, unique) 
    -- (intrusive_offset, unique + unique_disp) 
    -- (intrusive_offset + disp_eps, unique + unique_disp) 
    -- (intrusive_offset - disp_eps, unique + unique_disp) 
    -- (intrusive_offset, unique + unique_disp)
    -- (intrusive_offset, unique - unique_disp)
    -- (intrusive_offset + disp_eps, unique - unique_disp) 
    -- (intrusive_offset - disp_eps, unique - unique_disp) 
    -- (intrusive_offset, unique - unique_disp)
, gray);

real label_delta = 0.05 * max(max(linked, shared), max(intrusive, unique));
label("linked\underline{\hspace{0.3cm}}ptr", (linked_offset - label_offset, linked + label_delta), black);
label("std::shared\underline{\hspace{0.3cm}}ptr", (shared_offset - label_offset, shared + label_delta), black);
label("boost::intruisive\underline{\hspace{0.3cm}}ptr", (intrusive_offset - label_offset, intrusive + label_delta), black);
label("std::unique\underline{\hspace{0.3cm}}ptr", (unique_offset - label_offset, unique + label_delta), black);
label("copy constructor benchmark : ", (bench_width * 2.0, max_height * 1.1));
\end{asy}
\end{minipage}

\begin{
inked_offset, shared - shared_disp)
, gray);
draw(
    (shared_offset, intrusive) 
    -- (shared_offset, intrusive + intrusive_disp) 
    -- (shared_offset + disp_eps, intrusive + intrusive_disp) 
    -- (shared_offset - disp_eps, intrusive + intrusive_disp) 
    -- (shared_offset, intrusive + intrusive_disp)
    -- (shared_offset, intrusive - intrusive_disp)
    -- (shared_offset + disp_eps, intrusive - intrusive_disp) 
    -- (shared_offset - disp_eps, intrusive - intrusive_disp) 
    -- (shared_offset, intrusive - intrusive_disp)
, gray);
draw(
    (intrusive_offset, unique) 
    -- (intrusive_offset, unique + unique_disp)
    -- (intrusive_offset + disp_eps, unique + unique_disp) 
    -- (intrusive_offset - disp_eps, unique + unique_disp)
    -- (intrusive_offset, unique + unique_disp)
    -- (intrusive_offset, unique - unique_disp)
    -- (intrusive_offset + disp_eps, unique - unique_disp) 
    -- (intrusive_offset - disp_eps, unique - unique_disp) 
    -- (intrusive_offset, unique - unique_disp)
, gray);

real label_delta = 0.05 * max(max(linked, shared), max(intrusive, unique));
label("linked\underline{\hspace{0.3cm}}ptr", (linked_offset - label_offset, linked + label_delta), black);
label("std::shared\underline{\hspace{0.3cm}}ptr", (shared_offset - label_offset, shared + label_delta), black);
label("boost::intruisive\underline{\hspace{0.3cm}}ptr", (intrusive_offset - label_offset, intrusive + label_delta), black);
label("std::unique\underline{\hspace{0.3cm}}ptr", (unique_offset - label_offset, unique + label_delta), black);
label("std::move benchmark : ", (bench_width * 2.0, max_height * 1.1));
\end{asy}
\end{minipage}

\begin{minipage}[h]{0.49\linewidth}
\begin{asy}
real linked = 1187;
real shared = 3527;
real intrusive = 996;
real unique = 786;
real linked_disp = 5;
real shared_disp = 51;
real intrusive_disp = 18;
real unique_disp = 12;real bench_width = 1624.0;
real max_height = 3527;

real arrow_offset = bench_width / 4.0 * 1.1;
real point_offset = arrow_offset * 2 / 3;
real linked_offset = arrow_offset + bench_width;
real shared_offset = arrow_offset + 2 * bench_width;
real intrusive_offset = arrow_offset + 3 * bench_width;
real unique_offset = arrow_offset + 4 * bench_width;
real label_offset = (shared_offset - linked_offset) / 2.0;

size(20cm,0);
draw((arrow_offset / 2, 0) -- (arrow_offset / 2, max_height),arrow=Arrow);

label(
    scale(0.5) * ("" + string(linked) + ""),
    (point_offset, linked),
    gray
);
label(
    scale(0.5) * ("" + string(shared) + ""),
    (point_offset, shared),
    gray
);
label(
    scale(0.5) * ("" + string(intrusive) + ""),
    (point_offset, intrusive),
    gray
);
label(
    scale(0.5) * ("" + string(unique) + ""),
    (point_offset, unique),
    gray
);
path lin = box((arrow_offset,0), (linked_offset,linked));
fill(lin, deepgreen);

path sha = box((linked_offset,0), (shared_offset,shared));
fill(sha, royalblue);

path intr = box((shared_offset,0), (intrusive_offset,intrusive));
fill(intr, magenta);

path uniq = box((intrusive_offset,0), (unique_offset,unique));
fill(uniq, heavyred);

real disp_eps = bench_width / 20.0;
draw(
    (arrow_offset, linked) 
    -- (arrow_offset, linked + linked_disp) 
    -- (arrow_offset + disp_eps, linked + linked_disp) 
    -- (arrow_offset - disp_eps, linked + linked_disp) 
    -- (arrow_offset, linked + linked_disp)
    -- (arrow_offset, linked - linked_disp)
    -- (arrow_offset + disp_eps, linked - linked_disp) 
    -- (arrow_offset - disp_eps, linked - linked_disp) 
    -- (arrow_offset, linked - linked_disp)
, gray);
draw(
    (linked_offset, shared) 
    -- (linked_offset, shared + shared_disp) 
    -- (linked_offset + disp_eps, shared + shared_disp) 
    -- (linked_offset - disp_eps, shared + shared_disp) 
    -- (linked_offset, shared + shared_disp)
    -- (linked_offset, shared - shared_disp)
    -- (linked_offset + disp_eps, shared - shared_disp) 
    -- (linked_offset - disp_eps, shared - shared_disp) 
    -- (linked_offset, shared - shared_disp)
, gray);
draw(
    (shared_offset, intrusive) 
    -- (shared_offset, intrusive + intrusive_disp) 
    -- (shared_offset + disp_eps, intrusive + intrusive_disp) 
    -- (shared_offset - disp_eps, intrusive + intrusive_disp) 
    -- (shared_offset, intrusive + intrusive_disp)
    -- (shared_offset, intrusive - intrusive_disp)
    -- (shared_offset + disp_eps, intrusive - intrusive_disp) 
    -- (shared_offset - disp_eps, intrusive - intrusive_disp) 
    -- (shared_offset, intrusive - intrusive_disp)
, gray);
draw(
    (intrusive_offset, unique) 
    -- (intrusive_offset, unique + unique_disp) 
    -- (intrusive_offset + disp_eps, unique + unique_disp) 
    -- (intrusive_offset - disp_eps, unique + unique_disp) 
    -- (intrusive_offset, unique + unique_disp)
    -- (intrusive_offset, unique - unique_disp)
    -- (intrusive_offset + disp_eps, unique - unique_disp) 
    -- (intrusive_offset - disp_eps, unique - unique_disp) 
    -- (intrusive_offset, unique - unique_disp)
, gray);

real label_delta = 0.03 * max(max(linked, shared), max(intrusive, unique));
label("linked\underline{\hspace{0.3cm}}ptr", (linked_offset - label_offset, linked + label_delta));
label("std::shared\underline{\hspace{0.3cm}}ptr", (shared_offset - label_offset, shared + label_delta));
label("boost::intruisive\underline{\hspace{0.3cm}}ptr", (intrusive_offset - label_offset, intrusive + label_delta));
label("std::unique\underline{\hspace{0.3cm}}ptr", (unique_offset - label_offset, unique + label_delta));
label("real usecase (decart treee) benchmark : ", (bench_width * 2.0, max_height * 1.1));
\end{asy}
\end{minipage}

% \begin{figure}[h!]
%  \begin{minipage}[h]{0.49\linewidth}{1.0\textwidth}
%   \centering
%   \includegraphics{alloc}
%      \label{fig:sub:subfigure1}
%  \end{minipage}
%  \caption{\textsl{Figure text.}}
%  \label{fig:whole_figure}
%  \end{figure}

