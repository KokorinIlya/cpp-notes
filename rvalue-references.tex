\section{Предыстория к теме rvalue-reference}
Здесь речь пойдет о некоторых узких моментах языка на период С++03, которые и послужили мотивацией к ввидению в язык rvalue-reference.
\subsection{Что можно положить в вектор?}
Какие требования к типу {\bf T}, чтобы можно было объявить \mintinline{c++}{vector<T>} и потом этим пользоваться:
\begin{enumerate}
\item CopyConstructor - можно копировать
\item CopyAssignable - можно присваивать
\item Destroyble - можно разрушать
\end{enumerate}
Получается, что этим условиям не удовлетворяют: reference, const T, stream, RAII-classes\footnote{ RAII (resource acquisition is initialization) - идиома управления ресурсами, когда объект "захватывает" ресурс, то есть только он может к нему обращаться.}


Как поступать в таких случаях? Можно так:
\begin{enumerate}
\item хранить указатель => но тогда нужно иметь дело с динамической памятью и неудобным синтаксисом, и вообще это очень плохо:

\begin{minted}[
linenos,
frame=lines,
framesep=2mm,
breaklines]
{c++}

vec.push_back(new T(...)) // если при создании будет ошибка от resize() при выделении памяти, то объект не удалится

\end{minted}

\item unique\_ptr => это не плохой вариант
\item shared\_ptr => неудобный синтаксис \textbf{vec.push\_back(shared\_ptr<file\_descriptor>(new file\_descriptor()))}
\begin{minted}[
linenos,
frame=lines,
framesep=2mm]
{c++}

vec.push_back(shared_ptr<file_descriptor>(new file_descriptor(...)))

\end{minted}
\end{enumerate}

\subsection{Когда необходимо копирование?}
В С++ мы можем принимать аргументы функции по значению или по ссылке. И возвращать также. Меняется то, передаем ли мы сам объект или только его копию(значение). То есть если мы не хотим, чтобы объект менялся, то мы передаем его по значению, иначе по ссылке.


Ситуация:
\begin{minted}[linenos, frame=lines, framesep=2mm, tabsize = 4, breaklines]{c++}

void f(T); // получение по значению
T g(); // возврат по значению
int main() {
    f(g()); // здесь могут быть лишние копирования.
}
\end{minted}


Возникает вопрос: \textbf{Зачем нам копировать временные объекты, к которым все равно никто не может обратиться?}


Поэтому иногда компилятор творит магию. Обозначение в таблице:
\begin{enumerate}
    \item В таблице приведен код на C++, а рядом его эквивалент в языке C.
    \item  void* - это указатель на неинициализированную память, инача T*
\end{enumerate}
\begin{table}[H]
\begin{tabular}{|p{0.5\textwidth}|p{0.5\textwidth}|}
    \hline
    \multicolumn{2}{|p{0.5\textwidth}|}{Простой пример} \\
    \hline
    \begin{minted}[linenos, tabsize = 4, breaklines]{c++}
    // C++
    T obj(1, 2, 3);
    \end{minted}
    &
    \begin{minted}[linenos, tabsize = 4, breaklines]{c}
     // C
     char obj_storage[sizeof(T)];
     char_T(&obj_storage, 1, 2, 3);
    \end{minted}
    \\
    \hline
    \multicolumn{2}{|p{0.5\textwidth}|}{copy\_edision}  \\
    \hline
    \begin{minted}[linenos, tabsize = 4, breaklines]{c++}
    // C++
    T obj = g(); // => T obj(g());
    \end{minted}
    &
    \begin{minted}[linenos, tabsize = 4, breaklines]{c}
     // C
     char obj_storage[sizeof(T)];
     char_T(&obj_storage, obj);
    \end{minted}
    \\
    \hline
    \multicolumn{2}{|p{0.5\textwidth}|}{return\_value\_optimization (RVO)}  \\
    \hline
    \begin{minted}[linenos, tabsize = 4, breaklines]{c++}
    // C++
    T g() {
        return T(1, 2, 3);
    }
    \end{minted}
    &
    \begin{minted}[linenos, tabsize = 4, breaklines]{c}
     // C
     void g(void *result) {
         char_T(result, 1, 2, 3);
     }
    \end{minted}
    \\
    \hline
    \multicolumn{2}{|p{0.5\textwidth}|}{name\_return\_value\_optimization (NRVO)}  \\
    \hline
    \begin{minted}[linenos, tabsize = 4, breaklines]{c++}
    // C++
    void g() {
        T temp(1, 2, 3);
        temp.push_back(/*...*/);
        return temp;
    }
    \end{minted}
    &
    \begin{minted}[linenos, tabsize = 4, breaklines]{c}
     // C
     void g(void *result) {
         char_T(result, 1, 2, 3);
         ((*T)result)->push_back(/*...*/);
     }
    \end{minted}
    \\
    \hline
\end{tabular}
\end{table}


\textcolor{red}{NB}) Из таких оптимизаций, мы избегаем лишних копирований. Но могут быть и отрицательные эффекты: например нам теперь запрещено даже думать об глобальных побочных эффектах от конструкторов копирования.


\textcolor{red}{NB}) Может быть проблема с исключениями: ??? %TODO
