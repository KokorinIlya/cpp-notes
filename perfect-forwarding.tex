\section{Задача Perfect forwarding}

\subsection{Формулировка проблемы, попытки тривиального решения}

Пусть имеется функция \mintinline{c++}{g}, принимающая параметр типа \mintinline{c++}{T}. Стоит задача написать функцию \mintinline{c++}{f}, которая примет тот же параметр типа \mintinline{c++}{T}, сделает с ним что-нибудь (пусть в нижеследующем примере, без ограничения общности, \mintinline{c++}{f} не делает ничего) и вызвать функцию \mintinline{c++}{g} с этим аргументом.

Примером пары таких \mintinline{c++}{f} и \mintinline{c++}{g} могут служить \mintinline{c++}{std::make_unique<T>(params)} и конструктор \mintinline{c++}{T}.

\mintinline{c++}{make_unique} сначала вызовет конструктор \mintinline{c++}{T} с заданными параметрами, а потом обернёт получившийся объект в \mintinline{c++}{unique_ptr}. В процессе передачи параметров объектов в конструктор требуется так называемый perfect forwarding.

Первый вариант, который приходит на ум, выглядит следующим образом:

\begin{minted}[linenos, frame=lines, framesep=2mm, tabsize = 4, breaklines]{c++}
template <typename T>
void f(T x)
{
	g(x);
}
\end{minted}

Этот способ имеет недостатки. Рассмотрим случай, при котором \mintinline{c++}{g} принимает параметры по неконстантной ссылке и меняет их. Тогда

При вызове функции \mintinline{c++}{f} параметр \mintinline{c++}{x} будет проинициализирован копией аргумента вызова

В \mintinline{c++}{g} будет передана ссылка на копию исходного аргумента

В \mintinline{c++}{g} произойдёт изменение копии исходного аргумента, исходный аргумент же останется неизменным.

Это не то поведение, которого мы ожидаем, так как \mintinline{c++}{g} должна изменить наш аргумент.

Другое решение: заставить \mintinline{c++}{f} принимать параметр по константной ссылке

\begin{minted}[linenos, frame=lines, framesep=2mm, tabsize = 4, breaklines]{c++}
template <typename T>
void f(T const& x)
{
	g(x);
}
\end{minted}

При этом остаётся почти та же самая проблема: если \mintinline{c++}{g} принимает параметр по неконстантной ссылке, вызов будет невозможен (так как неконстантная ссылка не может быть инициализирована константной ссылкой)

Третье решение: нужно заставить \mintinline{c++}{f} принимать параметры по неконстантной ссылке

\begin{minted}[linenos, frame=lines, framesep=2mm, tabsize = 4, breaklines]{c++}
template <typename T>
void f(T& x)
{
	g(x);
}
\end{minted}

Это будет проблемой при попытке вызвать \mintinline{c++}{f} для константных объектов или для rvalue, то есть вызовы \mintinline{c++}{f(5)} и \mintinline{c++}{f(some_function())} станут невозможны.

Ни одно из приведённых выше решений не является универсальным. Возможно ли написать функцию \mintinline{c++}{f}, работающую во всех случаях?

Можно сделать две перегрузки \mintinline{c++}{f}: для константных и неконстантных ссылок.

\vspace{\baselineskip}

\begin{minted}[linenos, frame=lines, framesep=2mm, tabsize = 4, breaklines]{c++}
template <typename T>
void f(T const& x)
{
	g(x);
}

template <typename T>
void f(T& x)
{
	g(x);
}
\end{minted}

На первый взгляд это решение кажется хорошим, но очень скоро становится понятно, что тогда в \mintinline{c++}{f} нужно будет написать в два раза больше кода. К тому же, если \mintinline{c++}{g} и \mintinline{c++}{f} принимают по два параметра, придётся писать уже четыре перегрузки:

\begin{minted}[linenos, frame=lines, framesep=2mm, tabsize = 4, breaklines]{c++}
template <typename T>
void f(T& a, T& b)

template <typename T>
void f(T const& a, T& b)

template <typename T>
void f(T& a, T const& b)

template <typename T>
void f(T const& a, T const& b)
\end{minted}

для трёх параметров - 8, а для n - $2^n$. Поэтому этот способ не применим для большого количества аргументов.

Если же функция принимает неизвестное заранее количество параметров (например, в роли функции \mintinline{c++}{g} выступает конструктор неизвестного заранее типа \mintinline{c++}{T}, а в роли функции \mintinline{c++}{f} - функция \mintinline{c++}{make_unique}, как в примере выше, то такой код просто не может быть написан, так как на момент написания кода неизвестно количество параметров функций, а следовательно, неизвестно и количество необходимых перегрузок).

Прежде чем рассматривать грамотное C++ 11 решение этой проблемы, необходимо ближе познакомиться с правилами вывода ссылок в C++ 11.


\subsection{reference collapsing rule}

Как известно, в C++ не существует ссылок на ссылки. В тех случаях, когда возникает такой тип (в параметра шаблона или в псевдонимах типов), ссылки схлопываются, то есть происходит reference collapsing, и получается просто одна ссылка. Например:

\begin{minted}[linenos, frame=lines, framesep=2mm, tabsize = 4, breaklines]{c++}
template <typename T> 
f(T & x);

int x = 5;
f<int&>(x);
\end{minted}

В качестве \mintinline{c++}{T} подставляется \mintinline{c++}{int&}. При этой подстановке вместо \mintinline{c++}{T&},  должна была бы получится ссылка на ссылку \mintinline{c++}{int& &}, но происходит reference collapsing и получается просто \mintinline{c++}{int&}. Следовательно сигнатура функции \mintinline{c++}{f} после подстановки выглядит следующим образом: \mintinline{c++}{f<int&>(int& x)}.

Неформально правило reference collapsing в C++03 можно записать в следующем виде \mintinline{c++}{& + & = &}

С появлением в C++11 rvalue-ссылок потребовалось доопределить правила reference collapsing для них. Эти правила могут быть неформально записаны как “одиночный амперсант всегда побеждает”. Таким образом, таблица reference collapsing в C++ 11 выглядит так:

\mintinline{c++}{& + & = &}

\mintinline{c++}{& + && = &}

\mintinline{c++}{&& + & = &}

\mintinline{c++}{&& + && = &&}

Приведём несколько примеров этого правила:

\begin{center}
	
	\begin{tabular}{p{0.4\linewidth}p{0.4\linewidth}}
		\begin{minted}[linenos, frame=lines, framesep=2mm, tabsize = 0, breaklines]{c++}
		template <typename T>
		void f(T& x);
		
		int y;
		f<int&&>(y);
		\end{minted}
		&
		\begin{minted}[linenos, frame=lines, framesep=2mm, tabsize = 0, breaklines]{c++}
		template <typename T>
		void f(T&& x);
		
		f<int&&>(some_function());
		//some_function возвращает int
		\end{minted}
	\end{tabular}
	
	Тогда при инстанцировании шаблона вместо \mintinline{c++}{T} подставляется \mintinline{c++}{int&&}, сигнатура функции имеет вид
	
	\begin{tabular}{p{0.4\linewidth}p{0.4\linewidth}}
		\begin{minted}[linenos, frame=lines, framesep=2mm, tabsize = 0, breaklines]{c++}
		void f<int&&>(int&& & x);
		\end{minted}
		&
		\begin{minted}[linenos, frame=lines, framesep=2mm, tabsize = 0, breaklines]{c++}
		void f<int&&>(int&& && x);
		\end{minted}
	\end{tabular}
	
	После reference collapsing она превращается в:
	
	\begin{tabular}{p{0.4\linewidth}p{0.4\linewidth}}
		\begin{minted}[linenos, frame=lines, framesep=2mm, tabsize = 0, breaklines]{c++}
		void f<int&&>(int&& x);
		\end{minted}
		&
		\begin{minted}[linenos, frame=lines, framesep=2mm, tabsize = 0, breaklines]{c++}
		void f<int&&>(int&& x);
		\end{minted}
	\end{tabular}
\end{center}

\subsection{Правила особого вывода ссылок}

Пусть есть конструкция

\begin{minted}[linenos, frame=lines, framesep=2mm, tabsize = 4, breaklines]{c++}
template <typename T>
void func(T&& x);
\end{minted}

Пусть есть вызов \mintinline{c++}{func(expression)}, \mintinline{c++}{expression} имеет тип \mintinline{c++}{E}. Тогда если \mintinline{c++}{expression} является lvalue, \mintinline{c++}{T} выводится как \mintinline{c++}{E&}, если же \mintinline{c++}{expression} является rvalue, то \mintinline{c++}{T} выводится как \mintinline{c++}{E}\footnote{%
	Отметим, что переменная типа \mintinline{c++}{T&&}, использованная в функции выше, не всегда является rvalue-ссылкой, и называть её rvalue-ссылкой некорректно. В книге Скота Майерса “Эффективный и современныи С++” эти ссылки названы универсальными. Причины такого наименования станут ясны далее.}.

Примеры:

\begin{center}
	\begin{tabular}{p{0.4\linewidth}p{0.4\linewidth}}
		lvalue: & rvalue:\\
	\end{tabular}
	
	\begin{tabular}{p{0.4\linewidth}p{0.4\linewidth}}
		\mintinline[linenos, frame=lines, framesep=2mm, tabsize = 4, breaklines]{c++}{double pi = 3.14;}
		& \vspace{\baselineskip}\\
	\end{tabular}
	
	\begin{tabular}{p{0.4\linewidth}p{0.4\linewidth}}
		\mintinline{c++}{func(pi);}
		&
		\mintinline{c++}{func(4);}\\
	\end{tabular}
	
	\begin{tabular}{p{0.4\linewidth}p{0.4\linewidth}}
		\mintinline{c++}{pi} - lvalue типа \mintinline{c++}{double}, \mintinline{c++}{T} выводится как \mintinline{c++}{double&}, сигнатура \mintinline{c++}{func} имеет вид:
		&
		4 - rvalue типа \mintinline{c++}{int}, тогда \mintinline{c++}{T} выводится как \mintinline{c++}{int}, сигнатура \mintinline{c++}{func} имеет вид: \\
	\end{tabular}
	
	\begin{tabular}{p{0.4\linewidth}p{0.4\linewidth}}
		\begin{minted}[linenos, frame=lines, framesep=2mm, tabsize = 0, breaklines]{c++}
		void func<double&>(double& &&);
		\end{minted}
		&
		\begin{minted}[linenos, frame=lines, framesep=2mm, tabsize = 0, breaklines]{c++}
		void func<int>(int&&);
		\end{minted} 
		\\
	\end{tabular}
	
	\begin{tabular}{p{0.4\linewidth}p{0.4\linewidth}}
		после reference collapsing она приобретает вид: & \vspace{\baselineskip} \\
	\end{tabular}
	
	\begin{tabular}{p{0.4\linewidth}p{0.4\linewidth}}
		\begin{minted}[linenos, frame=lines, framesep=2mm, tabsize = 0, breaklines]{c++}
		void func<double&>(double&);
		\end{minted}
		& \vspace{\baselineskip} \\
	\end{tabular}
	
\end{center}


\vspace{\baselineskip}

Эти правила применяются для реешения проблемы perfect forwarding.

\subsection{Решение проблемы perfect forwarding}

Рассмотрим код способа передачи аргумента в \mintinline{c++}{g}:

\begin{minted}[linenos, frame=lines, framesep=2mm, tabsize = 4, breaklines]{c++}
template <typename T>
void f(T&& x)
{
	//Как можно передать x в g?
	g(x); //передать как lvalue
	g(std::move(x)); //передать как rvalue
}
\end{minted}

Ни один ни другой способ не подходит. \mintinline{c++}{x} должно передаваться так же, как получается, то есть поставлена задача сохранения value-category: пришедшее как rvalue значение, должно быть передано как rvalue, а пришедшее как lvalue должно быть передано как lvalue.

Рассмотрим вспомогательную функцию, с помощью которой эта задача будет выполняться.

\begin{minted}[linenos, frame=lines, framesep=2mm, tabsize = 4, breaklines]{c++}
template <typename U>
U&& forward(U& a)
{
	return static_cast<U&&>(a);
}
\end{minted}

Тогда код, использующий её, будет выглядеть так

\begin{minted}[linenos, frame=lines, framesep=2mm, tabsize = 4, breaklines]{c++}
template <typename T>
void f(T&& a)
{
	g(forward<T>(a));
}
\end{minted}

Данная функция реализует perfect forwarding. Чтобы убедиться в этом рассмотрим процесс вызова f для rvalue и lvalue и посмотрим на процесс вывода и подстановки аргументов шаблона.

\begin{center}
	\begin{tabular}{p{0.4\linewidth}p{0.4\linewidth}}
		lvalue: & rvalue:\\
	\end{tabular}
	
	\begin{tabular}{p{0.4\linewidth}p{0.4\linewidth}}
		\mintinline{c++}{int a}; & \vspace{\baselineskip}\\
	\end{tabular}
	
	\begin{tabular}{p{0.4\linewidth}p{0.4\linewidth}}
		\mintinline{c++}{f(a)}; & \mintinline{c++}{f(42)};\\
	\end{tabular}
	
	\begin{tabular}{p{0.4\linewidth}p{0.4\linewidth}}
		\mintinline{c++}{a} - lvalue типа \mintinline{c++}{int}, поэтому \mintinline{c++}{T} выводится в \mintinline{c++}{int&}, тогда после подстановки, до применения reference collapsing \mintinline{c++}{f} выглядит так: & \mintinline{c++}{42} - rvalue типа \mintinline{c++}{int}, следовательно \mintinline{c++}{Т} выводится в \mintinline{c++}{int}, а после подстановки \mintinline{c++}{f} выглядит так:\\
	\end{tabular}
	
	\begin{tabular}{p{0.4\linewidth}p{0.4\linewidth}}
		\begin{minted}[linenos, frame=lines, framesep=2mm, tabsize = 0, breaklines]{c++}
		void f<int&>(int& && a)
		{
		    g(forward<int&>(a));
		}
		\end{minted}
		&
		\begin{minted}[linenos, frame=lines, framesep=2mm, tabsize = 0, breaklines]{c++}
		void f<int>(int&& a)
		{
		    g(forward<int>(a));
		}
		\end{minted}
	\end{tabular}
	
	\begin{tabular}{p{0.4\linewidth}p{0.4\linewidth}}
		После применения reference collapsing, она будет выглядеть так:
		& \vspace{\baselineskip}\\
	\end{tabular}
	
	\begin{tabular}{p{0.4\linewidth}p{0.4\linewidth}}
		\begin{minted}[linenos, frame=lines, framesep=2mm, tabsize = 0, breaklines]{c++}
		void f<int&>(int& a)
		{
		    g(forward<int&>(a));
		}
		\end{minted}
		& \vspace{\baselineskip}\\
	\end{tabular}
	
	\begin{tabular}{p{0.4\linewidth}p{0.4\linewidth}}
		В \mintinline{c++}{forward} в качестве \mintinline{c++}{U} передают \mintinline{c++}{int&} (явно, в треугольных скобочках). Таким образом после подстановки \mintinline{c++}{forward} выглядит так:
		& В \mintinline{c++}{forward} в качестве \mintinline{c++}{U} передают \mintinline{c++}{int} (явно, в треугольных скобочках), и после подстановки \mintinline{c++}{forward} выглядит так:
		\\
	\end{tabular}
	
	\begin{tabular}{p{0.4\linewidth}p{0.4\linewidth}}
		\begin{minted}[linenos, frame=lines, framesep=2mm, tabsize = 0, breaklines]{c++}
		int& && forward<int&>(int& & a)
		{
		    return 
		        static_cast<int& &&>(a);
		}
		\end{minted}
		& \begin{minted}[linenos, frame=lines, framesep=2mm, tabsize = 0, breaklines]{c++}
		int&& forward<int>(int& a)
		{
		    return static_cast<int&&>(a);
		}
		
		\end{minted}
	\end{tabular}
	
	\begin{tabular}{p{0.4\linewidth}p{0.4\linewidth}}
		После reference collapsing сигнатура функции принимает вид: & \vspace{\baselineskip}\\
	\end{tabular}
	
	\begin{tabular}{p{0.4\linewidth}p{0.4\linewidth}}
		\begin{minted}[linenos, frame=lines, framesep=2mm, tabsize = 0, breaklines]{c++}
		int& forward<int&>(int& a)
		{
		    return static_cast<int&>(a);
		}
		\end{minted}
		& \vspace{\baselineskip}\\
	\end{tabular}
	
	\begin{tabular}{p{0.4\linewidth}p{0.4\linewidth}}
		Заметим, что эта функция не делает ничего, а просто кастит \mintinline{c++}{int&} к \mintinline{c++}{int&}. Поскольку \mintinline{c++}{forward} возвращает \mintinline{c++}{int&}, то результат его вызова - lvalue.Значит \mintinline{c++}{f} передаёт свой аргумент в \mintinline{c++}{g} как lvalue. Заметим, что изначально в \mintinline{c++}{f} a пришла как lvalue. Значит, для lvalue передача работает корректно. & Заметим, что в данном случае функция \mintinline{c++}{forward} работает как \mintinline{c++}{move}(приводит lvalue-ссылку к rvalue-ссылке). Результат вызова функции \mintinline{c++}{forward} является \mintinline{c++}{int&&}, а \mintinline{c++}{int&&}, возвращённая из функции, является rvalue, то есть результат применения \mintinline{c++}{forward} является rvalue. То есть \mintinline{c++}{f} передаёт \mintinline{c++}{a} как rvalue. Заметим, что получала \mintinline{c++}{f} этот параметр тоже как rvalue, значит, для rvalue передача работает корректно.\\
	\end{tabular}
	
\end{center}

Итого, удалось сохранить value-category. \mintinline{c++}{forward} работает либо как ничего (если ей передано lvalue), либо как \mintinline{c++}{move}(если ей передано rvalue).

\subsection{Одна частая ошибка при использовани forward}
Иногда при вызове \mintinline{c++}{forward} забывают указать параметр шаблона в треугольных скобках. Выглядит это так:

\begin{minted}[linenos, frame=lines, framesep=2mm, tabsize = 4, breaklines]{c++}
template <typename T>
void f(T&& a)
{
	g(forward(a));
}


\end{minted}

Покажем, почему это ошибка и как от неё избавиться. Расмотрим вызовы для lvalue и rvalue.

\begin{center}
	\begin{tabular}{p{0.4\linewidth}p{0.4\linewidth}}
		lvalue: & rvalue:\\
	\end{tabular}
	
	\begin{tabular}{p{0.4\linewidth}p{0.4\linewidth}}
		\mintinline{c++}{int a}; & \vspace{\baselineskip}\\
	\end{tabular}
	
	\begin{tabular}{p{0.4\linewidth}p{0.4\linewidth}}
		\mintinline{c++}{f(a);} & \mintinline{c++}{f(42);}\\
	\end{tabular}
	
	\begin{tabular}{p{0.4\linewidth}p{0.4\linewidth}}
		\mintinline{c++}{a} - lvalue типа \mintinline{c++}{int}, поэтому \mintinline{c++}{T} выводится в \mintinline{c++}{int&}, тогда после подстановки, до применения reference collapsing \mintinline{c++}{f} выглядит так: & 42 - rvalue типа \mintinline{c++}{int}, следовательно \mintinline{c++}{T} выводится в \mintinline{c++}{int}, а после подстановки \mintinline{c++}{f} выглядит так:\\
	\end{tabular}
	
	\begin{tabular}{p{0.4\linewidth}p{0.4\linewidth}}
		\begin{minted}[linenos, frame=lines, framesep=2mm, tabsize = 0, breaklines]{c++}
		void f(int& && a)
		{
		    g(forward(a));
		}
		\end{minted}
		&
		\begin{minted}[linenos, frame=lines, framesep=2mm, tabsize = 0, breaklines]{c++}
		void f<int>(int&& a)
		{
		    g(forward(a));
		}
		
		\end{minted}
	\end{tabular}
	
	\begin{tabular}{p{0.4\linewidth}p{0.4\linewidth}}
		После применения reference collapsing, она будет выглядеть так:
		& \vspace{\baselineskip}\\
	\end{tabular}
	
	\begin{tabular}{p{0.4\linewidth}p{0.4\linewidth}}
		\begin{minted}[linenos, frame=lines, framesep=2mm, tabsize = 0, breaklines]{c++}
		void f(int& a)
		{
		    g(forward(a));
		}
		\end{minted}
		& \vspace{\baselineskip}\\
	\end{tabular}
	
	\begin{tabular}{p{0.4\linewidth}p{0.4\linewidth}}
		Типы для \mintinline{c++}{forward} выводятся следующим образом:
		\mintinline{c++}{a} имеет тип \mintinline{c++}{int&}, а \mintinline{c++}{forward} принимает \mintinline{c++}{U&}, таким образом, \mintinline{c++}{U} выводится как \mintinline{c++}{int}. Тогда \mintinline{c++}{forward} выглядит следующим образом::
		& \mintinline{c++}{U} выводится как \mintinline{c++}{int&&}, тогда вместо \mintinline{c++}{U&} подставляется \mintinline{c++}{int&& &} (сжимается в \mintinline{c++}{int&&)}, а вместо \mintinline{c++}{U&&} подставляется \mintinline{c++}{int&& &&} (сжимается в \mintinline{c++}{int&&)} Тогда \mintinline{c++}{forward} выглядит следующим образом:
		\\
	\end{tabular}
	
	\begin{tabular}{p{0.4\linewidth}p{0.4\linewidth}}
		\begin{minted}[linenos, frame=lines, framesep=2mm, tabsize = 0, breaklines]{c++}
		int&& forward(int& a)
		{
		    return static_cast<int&&>(a);
		}
		
		\end{minted}
		& \begin{minted}[linenos, frame=lines, framesep=2mm, tabsize = 0, breaklines]{c++}
		int&& forward(int& a)
		{
		    return static_cast<int&&>(a);
		}
		
		
		\end{minted}
	\end{tabular}
	
	\begin{tabular}{p{0.4\linewidth}p{0.4\linewidth}}
		То есть \mintinline{c++}{forward} для lvalue работает как \mintinline{c++}{move}, а должен рабоать как ничего. & Таким образом, для rvalue \mintinline{c++}{forward} тоже сработает как \mintinline{c++}{move}.\\
	\end{tabular}
	
\end{center}

Получается что, если в \mintinline{c++}{forward} не указать параметр в треугольных скобках, то он всегда будет работать как \mintinline{c++}{move}.

Как можно решить эту проблему? Нужно запретить выводить тип для \mintinline{c++}{forward}, а позволить только явно указывать его. Рассмотрим вспомогательный класс.

\begin{minted}[linenos, frame=lines, framesep=2mm, tabsize = 4, breaklines]{c++}
template <typename T>
struct no_deduce
{
	typedef T type;
}
\end{minted}

Тогда \mintinline{c++}{forward} с использованием этого класса выглядит следующим образом:

\begin{minted}[linenos, frame=lines, framesep=2mm, tabsize = 4, breaklines]{c++}
template <typename U>
U&& forward(typename no_deduce<U>::type& a)
{
	return static_cast<U&&>(a);
}
\end{minted}

Заметим, что \mintinline{c++}{typename no_deduce<U>::type&} эквивалентно \mintinline{c++}{U&}.

Эта конструкция запрещает вывод типов, так как при передаче значения в функцию требуется вывести не \mintinline{c++}{U&}, а \mintinline{c++}{typename no_deduce<U>::type&}, а это невыводимый контекст. Компилятор может определить, каким типом должен быть \mintinline{c++}{typename no_deduce<U>::type} (пусть он должен иметь тип \mintinline{c++}{E}), но он не способен вывести тип \mintinline{c++}{U}, чтобы \mintinline{c++}{typename no_deduce<U>::type} имел тип \mintinline{c++}{E} (В данном конкретном случае это сделать можно (\mintinline{c++}{U} должен совпадать с \mintinline{c++}{E}), но в общем случае такая задача неразрешима, поэтому компилятор C++ не выводит типы в таких случаях).

Проблема решена.

Отметим напоследок, что \mintinline{c++}{forward} есть в стандартной библиотеке.

\subsection{Использование perfect forwarding}

Perfect forwarding используется, например, при использовании функций высшего порядка - то есть функций, принимающих другие функции в качестве аргументов или возвращающих их в качестве возвращаемого значения.

Без perfect forwarding, применение функций высшего порядка довольно обременительно, так как нет удобного способа передать аргументы в функцию внутри функции-обертки.
Возвращаясь к примеру, изложенному в начале главы, функцию-обёртку над конструктором типа \mintinline{c++}{T make_unique} с использованием perfect forwarding  можно реализовать следующим образом:

\begin{minted}[linenos, frame=lines, framesep=2mm, tabsize = 4, breaklines]{c++}
template <typename T, typename… Args>
//тип T - тип, обёртываемый в unique_ptr, Args - типы аргументов конструктора.
unique_ptr<T> make_unique(Args&&... args)
{
	return unique_ptr<T>(new T(std::forward<Args>(args)...))
}
\end{minted}

Про используемые в коде variadic template можно подробнее узнать в следующей главе.